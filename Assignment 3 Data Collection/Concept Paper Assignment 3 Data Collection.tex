\documentclass{article}
\usepackage[utf8]{inputenc}
\usepackage[T1]{fontenc}
\usepackage{geometry}
\geometry{a4paper}
\usepackage{helvet}
\renewcommand{\familydefault}{\sfdefault}
\title{A concept paper about social survey on the crimes experienced by members of society }
\author{Authored by Augustine Abule 15/U/2633/PS 215014371)}
\date{}
\setlength{\topmargin}{-1cm}
\begin{document}
\maketitle
\section{Abstract}
The purpose of this theory is to give a technological analysis of a data collection system designed to aid in carrying out 
a survey among members of civil society about the crimes they have experienced and how to better society. All this data is collected using smartphones \cite{key:1}. 
The system consists of two interconnected parts: a smartphone application ODK Collect \cite{key:2} that runs a data collection form, and an online aggregate
 server that receives the submitted data and compiles it in whatever format designed. 

This online data repository can then be mined in many ways possible to get a clearer understanding of the frequency of crime,
 what type of crime, most affected areas, most targeted gender, and much more categories in order to effectively sensitize the public, 
based on this data, and possibly reduce crime.

\section{Introduction}
Crime \cite{key:3} has always been a cancer in civil society, dating way back since the dawn of civilization. 
Crime is a very broad term, loosely meaning an illegal act or activity that can be punishable by law. Crimes can be divided into four major categories, personal crimes, property crimes, inchoate crimes and statutory crimes.

Crime, if not closely watched and fought, can be the downfall of civilization. It takes social sensitization and participation in the fight against this vice, and so through this project, I aim at getting people's personal accounts of their experiences of crimes in society. This data is very vital in tailoring useful and effective sensitization points and topics that will result into actual reduction in crime.

\section{Under-reported crime}
The Uganda police is the main government watchdog for crime in the country. They release annual reports regarding the frequencies 
and types of crimes nationwide. But there is growing concern in the public that these reports are not credible and accurate. 
This comes as no surprise since many individuals have lost the police’s trust to vigorously pursue their crime reports. 
And so an area that is portrayed by the police reports to have a relatively low crime rate may in fact have higher crime incidents since many go unreported.
 This can be misleading, especially for new residents planning a long term stay in that area.

\section{Using ODK as a fact checker} 
This concept aims at improving the situation by creating a very efficient way of carrying out a survey among permanent residents in a pilot area, 
get feedback from them regarding the crimes committed against them over a specified period of time. This collective data can then be analyzed and
 compared with the annual police reports regarding crime in that area. The public can then be sensitized accordingly, whether crime rates are actually 
being under reported, and measures they can take to secure their society.
	
Data collected in this survey ranges from the crimes committed against an individual, when they were committed, actions the individuals
 took after this experience, swiftness of police action upon reporting, and so much more.


\section{Muse and enthusiasm} 
There are already some non-governmental organizations that have come in to audit police reports and also carry out their 
surveys in order to come up with parallel reports. But from my experience, people may not entirely be open to employees of these
 NGOs about the crimes they have experienced. Some people may feel embarrassed to have been coned, or crimes like sexually assault
 may be too personal for them to share with a ‘stranger’. And so, many crimes again may go unreported.

Unlike the efforts of these NGOs, this system makes the individual anonymous and so they are given that confidence to be more open and
 frank about the crimes committed against them, in the hope that this information will help prevent the same from happening to other people, at the same time maintaining the privacy they demand.


\section{Design and Development}
This concept design and implementation is entirely based on the Open Data Kit (ODK) system of data collection. I have designed 
a form called ‘MY SOCIETY’ that runs on the ODK Collect mobile application. The form contains the survey questions that users can answer
 at their own convenience and then submit to the online aggregate server whose url is www.terror-snitch.appspot.com. I am then able to view
 all the various data provided by the people I have surveyed and analyze it in whatever fashion I desire in order to come up with conclusive statements and deductions based on this data.

\begin{thebibliography}{9}
\bibitem{key:1}
Ericsson. (2013, November) Ericsson Mobility Report. [Online].
http://www.ericsson.com/res/docs/2013/ericsson-mobility-report-november-2013.pdf
\bibitem{key:2}
https://opendatakit.org/use/collect/
\bibitem{key:3}
https://www.legalmatch.com/law-library/article/what-are-the-different-types-of-crimes.html

\end{document}
